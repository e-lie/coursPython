\PassOptionsToPackage{unicode=true}{hyperref} % options for packages loaded elsewhere
\PassOptionsToPackage{hyphens}{url}
%
\documentclass[]{article}
\usepackage[margin=2.5cm]{geometry}
\usepackage{lmodern}
\usepackage{amssymb,amsmath}
\usepackage{ifxetex,ifluatex}
\usepackage{fixltx2e} % provides \textsubscript
\ifnum 0\ifxetex 1\fi\ifluatex 1\fi=0 % if pdftex
  \usepackage[T1]{fontenc}
  \usepackage[utf8]{inputenc}
  \usepackage{textcomp} % provides euro and other symbols
\else % if luatex or xelatex
  \usepackage{unicode-math}
  \defaultfontfeatures{Ligatures=TeX,Scale=MatchLowercase}
\fi
% use upquote if available, for straight quotes in verbatim environments
\IfFileExists{upquote.sty}{\usepackage{upquote}}{}
% use microtype if available
\IfFileExists{microtype.sty}{%
\usepackage[]{microtype}
\UseMicrotypeSet[protrusion]{basicmath} % disable protrusion for tt fonts
}{}
\IfFileExists{parskip.sty}{%
\usepackage{parskip}
}{% else
\setlength{\parindent}{0pt}
\setlength{\parskip}{6pt plus 2pt minus 1pt}
}
\usepackage{hyperref}
\hypersetup{
            pdfborder={0 0 0},
            breaklinks=true}
\urlstyle{same}  % don't use monospace font for urls
\setlength{\emergencystretch}{3em}  % prevent overfull lines
\providecommand{\tightlist}{%
  \setlength{\itemsep}{0pt}\setlength{\parskip}{0pt}}
\setcounter{secnumdepth}{0}
% Redefines (sub)paragraphs to behave more like sections
\ifx\paragraph\undefined\else
\let\oldparagraph\paragraph
\renewcommand{\paragraph}[1]{\oldparagraph{#1}\mbox{}}
\fi
\ifx\subparagraph\undefined\else
\let\oldsubparagraph\subparagraph
\renewcommand{\subparagraph}[1]{\oldsubparagraph{#1}\mbox{}}
\fi

% set default figure placement to htbp
\makeatletter
\def\fps@figure{htbp}
\makeatother


\date{}

\begin{document}

\hypertarget{pruxe9nom-______________________}{%
\subsubsection{Prénom :
\_\_\_\_\_\_\_\_\_\_\_\_\_\_\_\_\_\_\_\_\_\_}\label{pruxe9nom-______________________}}

\hypertarget{nom-______________________}{%
\subsubsection{Nom :
\_\_\_\_\_\_\_\_\_\_\_\_\_\_\_\_\_\_\_\_\_\_}\label{nom-______________________}}

\hypertarget{initiation-uxe0-git-questionnaire}{%
\section{Initiation à Git :
questionnaire}\label{initiation-uxe0-git-questionnaire}}

\hypertarget{entourez-la-ou-les-bonnes-ruxe9ponses.}{%
\subsection{\texorpdfstring{Entourez la ou \emph{LES} bonnes
réponses.}{Entourez la ou LES bonnes réponses.}}\label{entourez-la-ou-les-bonnes-ruxe9ponses.}}

\hypertarget{question-1}{%
\subsubsection{Question 1}\label{question-1}}

Qu'est-ce qu'un commit ?

\begin{itemize}
\tightlist
\item
  A. Une étape validée du code qui apparaît dans l'historique du dépôt.
\item
  B. L'un des multiples historiques contenus dans un dépôt git.
\item
  C. N'importe quelle modification récente faite sur un des fichiers du
  dépôt.
\end{itemize}

\hypertarget{question-2}{%
\subsubsection{Question 2}\label{question-2}}

Comment connaître l'état courant d'un dépôt ?

\begin{itemize}
\tightlist
\item
  A. \texttt{git\ add}
\item
  B. \texttt{git\ checkout}
\item
  C. \texttt{git\ status}
\item
  D. \texttt{git\ branch}
\end{itemize}

\hypertarget{question-3}{%
\subsubsection{Question 3}\label{question-3}}

Qu'est-ce que HEAD ?

\begin{itemize}
\tightlist
\item
  A. Le serveur où on pousse son code.
\item
  B. Un curseur pointant sur un commit qu'on peut déplacer avec
  \texttt{git\ checkout}.
\item
  C. Une interface pour utiliser git.
\end{itemize}

\hypertarget{question-4}{%
\subsubsection{Question 4}\label{question-4}}

Une forge logicielle comme framagit est :

\begin{itemize}
\tightlist
\item
  A. Une plateforme alternative qui remplace git.
\item
  B. Une plateforme pour partager du code en ligne.
\item
  C. Une plateforme de tutoriels pour apprendre la programmation.
\item
  D. Une plateforme utile pour collaborer en entreprise.
\end{itemize}

\hypertarget{question-5}{%
\subsubsection{Question 5}\label{question-5}}

Git permet de :

\begin{itemize}
\tightlist
\item
  A. Gérer plusieurs versions du code d'un logiciel/script
\item
  B. Corriger automatiquement du code
\item
  C. Explorer l'historique du code d'un logiciel
\item
  D. Obtenir de l'aide sur la syntaxe en Python
\end{itemize}

\hypertarget{question-6}{%
\subsubsection{Question 6}\label{question-6}}

Où sont cachés les versions précédentes des fichiers dans git ?

\begin{itemize}
\tightlist
\item
  A. Dans Gitlens.
\item
  B. Dans un dossier invisible .git pour chaque dépôt.
\item
  C. Dans le dossier /etc.
\end{itemize}

\hypertarget{question-7}{%
\subsubsection{Question 7}\label{question-7}}

Pour changer de branche on utilise:

\begin{itemize}
\tightlist
\item
  A. \texttt{git\ reflog\ \textless{}nom\ de\ la\ branche\textgreater{}}
\item
  B.
  \texttt{git\ checkout\ \textless{}nom\ de\ la\ branche\textgreater{}}
\item
  C. \texttt{git\ clone\ \textless{}nom\ de\ la\ branche\textgreater{}}
\end{itemize}

\hypertarget{question-8}{%
\subsubsection{Question 8}\label{question-8}}

Habituellement, quel est le nom de la branche principale d'un dépôt ?

\begin{itemize}
\tightlist
\item
  A. \texttt{master}
\item
  B. \texttt{feature}
\item
  C. \texttt{main}
\end{itemize}

\hypertarget{question-9}{%
\subsubsection{Question 9}\label{question-9}}

Une branche est :

\begin{itemize}
\tightlist
\item
  A. Une nouvelle modification ajoutée à un dépôt
\item
  B. Une ligne d'historique du dépôt
\item
  C. Une opération de fusion
\end{itemize}

\hypertarget{question-10}{%
\subsubsection{Question 10}\label{question-10}}

Comment savoir quelles modifications ont été apportées lors du dernier
commit ?

\begin{itemize}
\tightlist
\item
  A. Utiliser \texttt{git\ diff\ HEAD\ HEAD\textasciitilde{}1}.
\item
  B. Utiliser \textbf{gitLens} dans VSCode avec la vue historique.
\item
  C. Utiliser \emph{Thonny} pour debugger le code.
\end{itemize}

\hypertarget{question-11}{%
\subsubsection{Question 11}\label{question-11}}

Une merge request sert à :

\begin{itemize}
\tightlist
\item
  A. Créer une discussion avec des collègues sur le code d'une fonction.
\item
  B. Faciliter la vérification collaborative du code.
\item
  C. Corriger automatiquement le code de votre nouvelle fonction.
\item
  D. Tester automatiquement le code ajouté.
\end{itemize}

\end{document}
